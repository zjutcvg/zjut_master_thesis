%-------------------------------------------------
% FileName: chapt-5.tex
% Version: 0.1
% Date: 2023-06-25
% Description: 第5章
% Others: 
% History: origin
%------------------------------------------------- 


% 断页
% \clearpage
\chapter{系统实现与测试}

\section{系统实现}
介绍主要功能模块的编程实现以及系统的部署方法。

\section{系统测试}
阐述系统的测试技术、测试过程和测试结果。

完全手动完成的表格,如表\ref{tab:tab1}所示。
\begin{table}[htbp] % htbp 浮动优先级
	\centering  % 居中
	\caption{一个表格}  % 表格标题
	\label{tab:tab1}  % 用于引用的label
	% 字母的个数对应列数,|代表分割线
	% l代表左对齐,c代表居中,r代表右对齐
	\begin{tabular}{|c|c|c|c|}   
		\hline  % 表格的横线 
		1 & 2 & 3 & 4 \\  % 表格中的内容,用&分开,\\表示下一行
		\hline 
		0.1 & 0.2 & 0.3 & 0.4 \\
		\hline
	\end{tabular}
\end{table}

以下编辑器(TexStudio)的表格向导生成的表格,如表\ref{tab:tab2}所示。
\begin{table}[htbp]
	\centering  
	\caption{诗词曲}   
    \label{tab:tab2}  
    \begin{tabular}{|c|c|c|c|}
        \hline 
          & 唐诗 & 宋词 & 元曲 \\ 
        \hline 
        1 & 李白 & 苏轼 & 关汉卿 \\ 
        \hline 
        2 & 白居易 & 辛弃疾 & 马致远 \\ 
        \hline 
        3 & 杜甫 & 李清照 & 张可久 \\ 
        \hline 
        4 & 王维 & 陆游 & 张养浩 \\ 
        \hline 
        5 & 孟浩然 & 欧阳修 & 徐再思 \\ 
        \hline 
    \end{tabular}  
\end{table}

嵌套表格,如表\ref{tab:tab3}所示。
\begin{table}[H]
	\centering  
	\caption{嵌套表格}   
    \label{tab:tab3} 
    \begin{tabular}{|c|c|c|}
        \hline
        a & b & c \\ \hline
        a & \multicolumn{1}{@{}c@{}|}
        {\begin{tabular}{c|c}
            e & f \\ \hline
            e & f \\
        \end{tabular}}
        & c \\ \hline
        a & b & c \\ \hline
    \end{tabular}
\end{table}

控制列宽和行距的表格,如表\ref{tab:tab4}所示。
\begin{table}[H]
    \centering  
	\caption{控制列宽和行距的表格}   
    \label{tab:tab4}
    \renewcommand\arraystretch{1.8}
    \begin{tabularx}{14em}
        {|*{4}{>{\centering\arraybackslash}X|}}
        \hline
        A & B & C & D \\ \hline
        a & b & c & d \\ \hline
    \end{tabularx}
\end{table}

表格单行内容太长,直接换行,如表\ref{tab:tab5}所示。
\begin{table}[htbp]
	\centering  
	\caption{单行内容太长直接换行}   
    \label{tab:tab5}  
    \begin{tabular}{|c|c|c|c|}
        \hline 
          & 唐诗 & 宋词 & 元曲 \\ 
        \hline 
        1 & 李白李白李白 & 苏轼苏轼苏轼苏轼 & 关汉卿关汉卿关汉卿关汉卿 \\
         & 李白李白李白李白 & 苏苏轼苏轼轼 & 关汉卿关汉卿关汉卿 \\
        \hline 
        2 & 白居易 & 辛弃疾 & 马致远 \\ 
        \hline 
    \end{tabular}  
\end{table}