%-------------------------------------------------
% FileName: chapt-1.tex
% Version: 0.1
% Date: 2023-06-25
% Description: 第1章
% Others: 
% History: origin
%------------------------------------------------- 

% 断页
\clearpage
% 页码从1开始计数
\setcounter{page}{1} 
% 阿拉伯数字显示页码
\pagenumbering{arabic}

\chapter{绪论}

% 这里的\label是为了下面的交叉引用
\section{课题背景}\label{sec:background}
简要介绍本文的开发背景。明确说明哪些是别人已经做过的工作,哪些是自己要做的工作。

% 分段是通过空行来实现的
话说得远一点,正是因为很多人不接受休谟的这个观点,才使得文艺创作者们有各种花招可以玩。比如《黑客帝国》后两集里的招数:让观众怀疑反抗军的基地也是虚拟出来的。比如《盗梦空间》里,让观众怀疑所谓的真实世界还是一个梦境。

% 引用参考文献 [1]
叔本华认为,我们可以提高自己对这世界的认识(当然是去认识叔本华所理解的那个世界),把自己的感情和欲望上升为全人类的感情和欲望,这样就可以消除个人的欲望\cite{chen2005laser}。

% 引用多篇参考文献 [1-3]
贪婪是人的本性,也是资本主义社会发展的必不可少的动力。但是今天的资本主义社会学会了用很多方法去克制个人贪欲。比如通过宗教的约束,比如通过立法的形式,遏制垄断企业(可怜的微软),遏制不正当和不道德的竞争,给工人更多的福利\cite{chen2005laser, mittelbach2004latex, zhen2018leave}。

% 下面是一个有序列表的例子,默认编号
别人已经研究的工作包括:
\begin{enumerate}
	\item 古希腊的斯多葛学派就相信部分决定论。他们认为我们不能控制事物,但是可以控制我们自己对待生活的方式。所以这个学派提倡随遇而安的生活态度\cite{zhou2002nerualnet}。
	\item 斯宾诺莎是用类似于几何的逻辑一步步推出整个哲学体系的。这意味着,他相信世间万物之间都有着严格的逻辑关系。这必然也会导致决定论\cite{qi2020deeplearning}。
	\item 休谟认为他之前的经验主义者和理性主义者都存在根本缺陷。休谟的回答是,不知道就不知道,没关系。我们能得到的经验就是面前的生活,在有明确的证据证明面前的生活都是幻觉之前,我们就照着自己平时的经验正常生活下去就可以了。我们没必要也没能力去无限地怀疑世界\cite{partl2019short}。
\end{enumerate}

% 有序列表嵌套 定制编号
唐诗,宋词,元曲举例:
\begin{enumerate}
	\item 唐诗
	\begin{enumerate} 
		\item 蜀道难(李白)噫吁嚱,危乎高哉!蜀道之难,难于上青天!蚕丛及鱼凫,开国何茫然!尔来四万八千岁,不与秦塞通人烟。西当太白有鸟道,可以横绝峨眉巅。地崩山摧壮士死,然后天梯石栈相钩连。上有六龙回日之高标,下有冲波逆折之回川。黄鹤之飞尚不得过,猿猱欲度愁攀援。青泥何盘盘,百步九折萦岩峦。扪参历井仰胁息,以手抚膺坐长叹。
		\item 春晓(孟浩然)春眠不觉晓,处处闻啼鸟。夜来风雨声,花落知多少。
	\end{enumerate}
	\item 宋词
	\begin{enumerate}
		\item 破阵子(辛弃疾)醉里挑灯看剑,梦回吹角连营。八百里分麾下炙,五十弦翻塞外声,沙场秋点兵。马作的卢飞快,弓如霹雳弦惊。了却君王天下事,赢得生前身后名。可怜白发生!
		\item 赤壁怀古(苏轼)大江东去,浪淘尽,千古风流人物。故垒西边,人道是,三国周郎赤壁。乱石穿空,惊涛拍岸,卷起千堆雪。江山如画,一时多少豪杰。遥想公瑾当年,小乔初嫁了,雄姿英发。羽扇纶巾,谈笑间,樯橹灰飞烟灭。故国神游,多情应笑我,早生华发。人生如梦,一尊还酹江月。
	\end{enumerate}
	\item 元曲
	\begin{enumerate}
		\item 窦娥冤(关汉卿)花有重开日,人无再少年。不须长富贵,安乐是神仙。老身蔡婆婆是也。楚州人氏,嫡亲三口儿家属。不幸夫主亡逝已过,止有一个孩儿,年长八岁。俺娘儿两个,过其日月。家中颇有些钱财。这里一个窦秀才,从去年问我借了二十两银子,如今本利该银四十两。我数次索取,那窦秀才只说贫难,没得还我。他有一个女儿,今年七岁,生得可喜,长得可爱。我有心看上他,与我家做个媳妇,就准了这四十两银子,岂不两得其便!他说今日好日辰,亲送女儿到我家来。老身且不索钱去,专在家中等候。这早晚窦秀才敢待来也。
		\item 秋思(马致远)枯藤老树昏鸦,小桥流水人家,古道西风瘦马。夕阳西下,断肠人在天涯。
	\end{enumerate}
\end{enumerate}

% 这里的\label是为了下面的交叉引用
\section{目的意义}\label{sec:meaningful}
介绍本课题的研究意义、研究目的、主要研究内容、研究范围和应该解决的问题。

% 下面演示怎么增加子标题
\subsection{目的意义1}
\subsubsection{目的意义11}
\subsubsection{目的意义12}
\subsubsection{目的意义13}
\subsection{目的意义2}
\subsection{目的意义3}
\subsection{目的意义4}

\section{论文主要工作}
介绍本研究课题的来源及主要研究内容。

% 下面是一个引用节的例子
论文的背景见\ref{sec:background},论文的目的意义见\ref{sec:meaningful}。

本作品分工如下,虚若无同学实现:
% 下面是一个无序列表的例子
\begin{itemize}
	\item 系统架构设计;
	\item 功能模块的设计与实现;
	\item web端的编程与实现;
	\item 数据库设计。
\end{itemize}

欧阳潇潇同学实现:
\begin{itemize}
	\item 微信小程序的设计与实现;
	\item 微信端接口的实现;
	\item 数据库设计。
\end{itemize}

\section{论文组织结构}

第1章介绍了考研教室预约系统的课题背景,目的意义,组员分工,全文的组织结构。

第2章介绍了系统开发所涉及的相关技术。包括MySQl数据库,前后端分离。Spring Boot,Ajax,微信小程序开发。
 
第3章对考研教室预约系统做了详细的需求分析,详细介绍了系统在实际应
用中的功能需求,系统的业务分析,系统的用例分析,系统的功能分析和非功能性需求。

第4章考研教室预约系统的系统总体设计,详细介绍了系统的总体结构和系统模块的设计以及数据库的设计和E-R图。

第5章考研教室预约系统的系统的实现,介绍了系统实现的关键技术,以及WEB端和移动端各个功能模块的实现。

第6章总结和展望,总结了系统的开发工作,分析了系统目前存在的问题及系统需要进一步完善的地方。