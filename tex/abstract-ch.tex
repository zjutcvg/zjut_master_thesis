%-------------------------------------------------
% FileName: abstract-ch.tex
% Version: 0.1
% Date: 2023-06-25
% Description: 中文摘要
% Others: 
% History: origin
%------------------------------------------------- 

% 以下不用改动-------------------------------------
% 断页
\clearpage
% 页码从1开始计数
\setcounter{page}{1}
% 大写罗马数字显示页码
\pagenumbering{Roman}
% 加入书签, bm@abstractname要唯一
\addcontentsline{toc}{chapter}{摘要}
\currentpdfbookmark{\defabstractname}{bm@abstractname}
% \chapter*{} 表示不编号,不生成目录
% \markboth{}{} 用于页眉
% 此处以中文题目作为章题目
\chapter*{\defTitleCn\markboth{\defabstractname}{}}


% 修改摘要和关键词---------------------------------
% 中文摘要
\abstract{
摘要反映了毕业设计(论文)的主要信息,以浓缩的形式概括说明研究目的、内容、方法、成果和结论,具有独立性和完整性。中文摘要一般为400字左右,不含公式、图表和注释。论文摘要应采用第三人称的写法,力求文字精悍简练。

摘要要交代清楚毕业设计的几个问题:why,what,how,results和meaningful。Why为什么要设计这个作品,通常是为了解决某个问题,比如现有的设计有缺陷,或者用户有某些需要。What完成了一个什么样的作品,具备哪些功能。How怎么完成这个作品的,用了哪些技术,一些关键功能模块是怎么设计和实现的。Results和meaningful有什么结果和意义,通常去回答开始提出的why,作品确实解决了某个问题,或者取得了某个效果。
}

% 中文关键词
% 关键词是供检索用的主题词条,应采用能覆盖毕业设计(论文)主要内容的通用技术词条(参照相应的技术术语标准)。关键词一般为3~5个,每个关键词不超过5个字。
\keywords{毕业设计;作品;技术;结果;意义}


