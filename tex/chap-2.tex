%-------------------------------------------------
% FileName: chapt-2.tex
% Version: 0.1
% Date: 2023-06-25
% Description: 第2章
% Others: 
% History: origin
%------------------------------------------------- 

% 断页
% \clearpage
\chapter{相关技术和理论基础}

\section{技术与理论基础}
介绍在系统的开发过程中所要用到的技术\cite{timmurphy}以及与系统相关的理论知识\cite{huwei2017latex2e}。

\section{质能方程}
% 行内公式,用两个$
质能方程即描述质量与能量之间的当量关系的方程\cite{liuxiaopingwordandtex}。质能方程$e=mc^2$,$e$表示能量,$m$代表质量,而$c$则表示光速,由爱因斯坦提出\cite{yassin1994latex}。

\section{牛顿力学}
% 下面说明公式的引用。
任何物体都要保持匀速直线运动或静止状态\cite{liu2013latex},直到外力迫使它改变运动状态为止,见公式\eqref{eq:newton}。
% 行间公式,用环境 equation
% \label用于标注公式,在别处引用
\begin{equation}\label{eq:newton}
	\vec{F}=m\vec{a} 
\end{equation}

\section{勾股定理}
勾股定理是一个基本的几何定理,指直角三角形的两条直角边的平方和等于斜边的平方,如方程\eqref{eq:pythagoraslaw}所示。中国古代称直角三角形为勾股形,并且直角边中较小者为勾,另一长直角边为股,斜边为弦,所以称这个定理为勾股定理,也有人称商高定理\cite{he2017mask}。
\begin{equation}\label{eq:pythagoraslaw}
    a^2 + b^2 = c^2  
\end{equation}

\section{线性代数}
线性代数是数学的一个分支,它的研究对象是向量,向量空间(或称线性空间),线性变换和有限维的线性方程组。向量空间及其线性变换,以及与此相联系的矩阵(形如\eqref{eq:linearalgebra})理论,构成了线性代数的中心内容。
\begin{equation}\label{eq:linearalgebra}
	\begin{pmatrix}
		a_{11} & a_{12} & a_{13}\\ 
		a_{21} & a_{22} & a_{23}\\  
		a_{31} & a_{32} & a_{33}   
	\end{pmatrix}  
\end{equation}

\section{量子力学}
对于微观粒子的运动,可以用薛定谔方程来描述,
\begin{equation}
	\hat H \Psi = i \hbar \frac{\partial \Psi}{\partial t}
\end{equation}
其中$\hat H $为哈密顿算符,一般的从一个粒子的质量与这个粒子的势能函数,就可以得到这个方程,然后再根据给定的初值条件和边值条件,就可以解出我们需要的描述粒子运动状态的波函数来,然后波函数的绝对值平方就给出了粒子在一定时空位置的分布几率,这就是我们所能得到的关于粒子的最详尽的运动状态信息。

